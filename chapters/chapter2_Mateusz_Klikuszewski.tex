\section{Mateusz Klikuszewski}
\label{sec:matklik}

It's me(see Figure ~\ref{fig:mk}).

\begin{figure}[htbp]
    \centering
    \includegraphics[width=0.75\textwidth]{pictures/mk.jpg}
    \caption{My photo}
    \label{fig:mk}
\end{figure}

Table ~\ref{tab:tab2} contains multiplication table up to 5

\begin{table}[htbp]
\centering
\begin{tabular}{|
>{\columncolor[HTML]{FFFE65}}l |
>{\columncolor[HTML]{F8A102}}l 
>{\columncolor[HTML]{F8A102}}l 
>{\columncolor[HTML]{F8A102}}l 
>{\columncolor[HTML]{F8A102}}l 
>{\columncolor[HTML]{F8A102}}l }
\hline
\cellcolor[HTML]{F8A102}{\color[HTML]{333333} Lp.} & \multicolumn{1}{l|}{\cellcolor[HTML]{FE0000}{\color[HTML]{333333} \textbf{1}}} & \multicolumn{1}{l|}{\cellcolor[HTML]{FE0000}{\color[HTML]{333333} \textbf{2}}} & \multicolumn{1}{l|}{\cellcolor[HTML]{FE0000}{\color[HTML]{333333} \textbf{3}}} & \multicolumn{1}{l|}{\cellcolor[HTML]{FE0000}{\color[HTML]{333333} \textbf{4}}} & \multicolumn{1}{l|}{\cellcolor[HTML]{FE0000}{\color[HTML]{333333} \textbf{5}}} \\ \hline
\textbf{1}                                         & 1                                                                              & 2                                                                              & 3                                                                              & 4                                                                              & 5                                                                              \\ \cline{1-1}
\textbf{2}                                         & 2                                                                              & 4                                                                              & 6                                                                              & 8                                                                              & 10                                                                             \\ \cline{1-1}
\textbf{3}                                         & 3                                                                              & 6                                                                              & 9                                                                              & 12                                                                             & 15                                                                             \\ \cline{1-1}
\textbf{4}                                         & 4                                                                              & 8                                                                              & 12                                                                             & 16                                                                             & 20                                                                             \\ \cline{1-1}
\textbf{5}                                         & 5                                                                              & 10                                                                             & 15                                                                             & 20                                                                             & 25                                                                             \\ \cline{1-1}
\end{tabular}
\label{tab:tab2}
\caption{Multiplication table}
\end{table}

The most beautiful equation in mathematics:
\begin{math}e^{i\pi} + 1 = 0\end{math}

\subsection{Top 10 Roman Empires}

\begin{enumerate}
    \item Roman Empire
    \item Byzantine Empire
    \item Holy Roman Empire
    \item Ottoman Empire
    \item Frankish Empire
    \item Kingdom of Italy
    \item Russian Empire
    \item Bulgarian Empire
    \item Papal State
    \item Lechina Empire
\end{enumerate}


\subsection{Best HOI4 mods}
\begin{itemize}
    \item Kaiserredux
    \item Red Flood
    \item The Great War Redux
    \item Regnum Poloniae
    \item Rise of Nations
    \item Modern Day
    \item The New Order
    \item Equestria at war
\end{itemize}

\subsection{Przepis na bigos}
\begin{center}
    \textbf{PRZYGOTOWANIE}\\
    \textit{Mięso pokroić w kostkę. \emph{Cebulę} pokroić w kosteczkę i zeszklić na oleju w dużym garnku. \underline{Dodać mięso }i dokładnie je obsmażyć. Wlać \textbf{2} szklanki gorącego \emph{bulionu lub wody z solą i pieprzem}, zagotować. Następnie dodać połamane suszone \emph{grzyby}, przykryć, zmniejszyć ogień i gotować przez ok. \textbf{45} minut.} Dodać listek \textit{laurowy, ziela angielskie, kminek, majeranek, powidła śliwkowe lub posiekane śliwki}, obrane i pokrojone w kosteczkę obrane \textit{jabłko} i wymieszać. \underline{ Dodać odciśniętą kiszoną \emph{kapustę}} oraz wlać szklankę \textit{wody}, wymieszać. Przykryć i gotować przez ok. \textbf{15} minut. \textit{Kiełbasę} obrać ze skóry, pokroić w kostkę i podsmażyć na patelni. Dodać do kapusty i gotować przez ok. \textbf{30} minut.
    
    Pod koniec dodać koncentrat pomidorowy. Mąkę podsmażyć na suchej patelni, gdy zacznie brązowieć dodać łyżkę \textit{masła} i mieszać aż masło się rozpuści. Trzymając patelnię na ogniu dodać stopniowo kilka łyżek kapusty cały czas mieszając. Przełożyć zawartość patelni z powrotem do garnka, wymieszać i zagotować.

    Mateusz K (See Figure ~\ref{fig:mk})

\end{center}
