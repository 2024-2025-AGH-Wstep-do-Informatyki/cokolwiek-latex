\section{Modularna aplikacja do wykonywania obliczeń. }
\label{sec:github}

\subsection{Dlaczego?}
Projekt ten wybraliśmy ze względu na połączenie niskiego progu wejściowego z wieloma opcjami rozbudowy, podstawowa funkcja kalkulatora może zostać rozwinięta poprzez dodanie bardziej zaawansowanych funkcji takich jak m.in. przeliczanie jednostek, tworzenie grafów, znajdowanie pierwiastków wielomianu, itd. Projekt ten posiada szerokie możliwości rozwoju.

\subsection{UI}
Stworzymy minimalistyczny interfejs użytkownika pozwalający przełączać się między modułami programu, inspiracją do wyglądu programu może być podstawowa aplikacja kalkulatora w systemie Windows.

\subsection{Koncepty modułów}

Poniższa lista zawiera tylko pomysły na funkcje programu, bez obietnicy ich wprowadzenia.
\begin{itemize}
    \item Standardowy kalkulator
    \item Zaawansowany kalkulator (pierwiastki wyższych stopni, logarytmy, silnia, dwumian)
    \item Obsługa funkcji trygonometrycznych i cyklometrycznych
    \item Regnum Poloniae
    \item Konwerter jednostek
    \item Przeliczanie systemów liczbowych
    \item Tworzenie wykresów funkcji
    \item Wyszukiwanie pierwiastków wielomianu
    \item Liczenie pochodnych wielomianów
    \item Obliczanie w ułamkach dziesiętnych i zwykłych
    \item Pamięć obliczeń
    \item Możliwość cofania i edycji działań
    \item Notatnik matematyczny
    \item Możliwość dodawania modułów z własnymi wzorami
    \item Operacje na macierzach
    \item Operacje na wektorach
    \item Przekształcenia wzorów
    \item Operacje na liczbach zespolonych
    \item Arytmetyka modularna + NWW i NWD
\end{itemize}



















